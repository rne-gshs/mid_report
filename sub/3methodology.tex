\section{연구 내용}
\subsection{연구 방법}
C++ 프로그래밍 언어를 이용하여 블로토 대령 게임에서 어떤 배치와 무작위적으로 선정한 배치를 겨루는 것을 반복하여 해당 배치의 승률을 구하는 프로그램을 작성하였다. 이 프로그램은 이미 진행된 연구 뿐만 아니라 앞으로의 연구에도 지속적으로 사용될 것이다. 위 프로그램이 의도대로 구현되었는지 검증하고자 프로그램을 통해 출력되는 배치의 무작위성을 확인하였다. 프로그램을 통해 합이 20인 5개의 자연수를 출력하였으며, 총 20000번 프로그램을 실행하여 출력된 각 배치에 포함된 특정 숫자의 개수를 구하였다. 이후 이론값과 실제 데이터를 비교하여 $R^2$ 값을 구한 결과 $R^2=0.99989$로 프로그램이 무작위적으로 병력 배치를 선정하고 있음을 확인하였다.

\begin{figure}[htb!]
    \vspace{1em}
    \centering
    \pgfplotsset{width=0.8\textwidth, height=0.53\textwidth}
    \begin{tikzpicture}
    \begin{axis}[
        % title={프로그램의 무작위성 검증},
        xlabel={출력되는 자연수},
        ylabel={출력 횟수의 기댓값},
        xmin=0, xmax=25, % x축 범위
        ymin=0, ymax=0.9, % y축 범위
        xtick={0, 5, 10, 15, 20, 25},
        ytick={0, 0.1, 0.2, 0.3, 0.4, 0.5, 0.6, 0.7, 0.8, 0.9},
        legend pos=north west,
        xmajorgrids=true,
        ymajorgrids=true,
        grid style=dashed,
    ]
            \addplot[
                color=blue,
                mark=square,
                ]
                coordinates {
                    (0, 0.833)
                    (1, 0.725)
                    (2, 0.626)
                    (3, 0.536)
                    (4, 0.456)
                    (5, 0.384)
                    (6, 0.320)
                    (7, 0.264)
                    (8, 0.214)
                    (9, 0.171)
                    (10, 0.135)
                    (11, 0.104)
                    (12, 0.078)
                    (13, 0.056)
                    (14, 0.040)
                    (15, 0.026)
                    (16, 0.016)
                    (17, 0.009)
                    (18, 0.005)
                    (19, 0.002)
                    (20, 0.0005)
                };
                \addplot[
                color=orange,
                mark=square,
                ]
                coordinates {
                    (0, 0.831)
                    (1, 0.727)
                    (2, 0.626)
                    (3, 0.542)
                    (4, 0.453)
                    (5, 0.383)
                    (6, 0.312)
                    (7, 0.264)
                    (8, 0.218)
                    (9, 0.174)
                    (10, 0.135)
                    (11, 0.105)
                    (12, 0.076)
                    (13, 0.056)
                    (14, 0.040)
                    (15, 0.026)
                    (16, 0.016)
                    (17, 0.011)
                    (18, 0.004)
                    (19, 0.002)
                    (20, 0.0004)
                };
            \legend{실제값, 이론값}
    
    \end{axis}
    \end{tikzpicture}
    \caption{프로그램의 무작위성 검증}
    \label{fig:enter-label}
\end{figure}

\subsection{기존 블로토 대령 게임에서의 최적 전략 분석}

\subsubsection{상대의 선택이 완전히 무작위적인 경우}
각 병력 배치가 무작위하게 출력되는 다른 병력 배치와 겨루는 게임 수 $g=30000$으로 두어 병력 배치 별 승률을 계산하였다. 전장의 수 $n=5$, 각 참가자에게 주어진 병력의 수 $k=5a$으로 설정하여 자연수 $a$의 값에 따른 병력 배치별 승률의 경향성을 분석하였다. 그 결과 $1 \leq a \leq 5$ 범위의 모든 $a$에 대해 5개의 전장에 동일한 $a$명의 병력을 배치할 때의 승률이 가장 높게 나타났으며, 병력 배치의 표준편차가 낮을수록 높은 승률을 보이는 경향성을 확인하였다. 또한, $a$의 값이 증가함에 따라 5개의 전장에 동일한 병력을 배치할 때의 승률도 증가하였다.

\begin{figure}[htb!]
    \vspace{1em}
    \centering
    \pgfplotsset{width=0.8\textwidth, height=0.4\textwidth}
    \begin{tikzpicture}
    \begin{axis}[
        % title={제목},
        xlabel={$a$값},
        ylabel={균등 분배 전략의 승률},
        xmin=0, xmax=6, % x축 범위
        ymin=0.44, ymax=0.9, % y축 범위
        xtick={1, 2, 3, 4, 5},
        ytick={0, 0.1, 0.2, 0.3, 0.4, 0.5, 0.6, 0.7, 0.8, 0.9},
        legend pos=north west,
        xmajorgrids=true,
        ymajorgrids=true,
        grid style=dashed,
    ]
            \addplot[
                color=red,
                mark=square,
                ]
                coordinates {
                    (1, 0.5972)
                    (2, 0.6403)
                    (3, 0.6606)
                    (4, 0.6745)
                    (5, 0.6957)
                };
                
    \end{axis}
    \end{tikzpicture}
    \caption{a값에 따른 균등 분배 전략 승률 그래프}
    \label{fig:enter-label}
\end{figure}

% //그래프 삽입(1)
% - a가 1에서 5일 때 x축 표준편차, y축 승률 그래프
% - a가 1에서 5일 때 병력 배치 (a, a, a, a, a)의 승률 그래프


위의 그래프는 상대의 선택이 완전히 무작위적인 경우에 병력을 가능한 균등하게 분배하는 배치가 가장 유리함을 시사한다. 이러한 배치는 전장의 수가 일정할 때 주어진 병력의 수가 증가할 수록 더욱 효과적임을 알 수 있다. 이러한 병력 배치 방법을 본 논문에서는 '균등 분배 전략'이라고 표현하였다.

\subsubsection{상대의 선택이 승률에 선형적으로 비례하는 경우}
III.2.1에서는 상대의 선택이 완전히 무작위적인 경우만을 고려했으나, 실제 환경에서는 상대가 무작위적으로 배치를 선택하는 것이 아니라 높은 승률의 배치를 선택하고자 할 것이다. 이러한 심리적 요인을 고려하기 위해 다음과 같은 과정을 진행하였다. 

1. 모든 병력 배치가 같은 개수만큼 존재하는 초기 set를 $S_0$라고 한다.

2. 모든 배치를 $S_0$의 임의의 배치와 겨루는 것을 반복하여 각 배치별 승률을 구한다.

3. 승률에 선형적으로 비례하게 각 배치가 선택될 확률을 조정한 set $S_1$을 얻는다. 

4. 같은 방법으로 $S_n$을 이용하여 $S_{n+1}$을 얻는 시행을 반복하여 최적 배치를 파악한다.

전장의 수 $n=5$, 각 참가자에게 주어지는 병력의 수 $k=20$, 게임 수 $g=50000$로 설정하여 반복 횟수 $1 \leq r \leq 5$ 범위에서 병력 배치 별 승률의 경향성을 분석하였다. 그 결과, 범위 내의 모든 $r$값에 대해 균등 분배 전략 (4, 4, 4, 4, 4)가 가장 높은 승률을 보였다. $r=1$일 때의 병력 배치의 표준 편차에 따른 승률과 $r$의 값에 따른 균등 분배 전략의 승률은 다음과 같다.

\begin{figure}[htb!]
    \vspace{1em}
    \centering
    \pgfplotsset{width=0.8\textwidth, height=0.5\textwidth}
\begin{tikzpicture}
    \begin{axis}[
        % title={제목},
        xlabel={반복 횟수 $r$},
        ylabel={균등 분배 전략의 승률},
        xmin=0, xmax=6, % x축 범위
        ymin=0.36, ymax=0.79, % y축 범위
        xtick={1, 2, 3, 4, 5},
        ytick={0, 0.1, 0.2, 0.3, 0.4, 0.5, 0.6, 0.7, 0.8, 0.9},
        legend pos=north west,
        xmajorgrids=true,
        ymajorgrids=true,
        grid style=dashed,
    ]
            \addplot[
                color=green,
                mark=square,
                ]
                coordinates {
                    (1, 0.58842)
                    (2, 0.5606)
                    (3, 0.5644)
                    (4, 0.5658)
                    (5, 0.56776)
                };
    
    \end{axis}
    \end{tikzpicture}
    \caption{반복 횟수가 1에서 5로 변할 때 (4, 4, 4, 4, 4)의 승률 그래프}
    \label{fig:enter-label}
\end{figure}

% //그래프 삽입(2)
% - 반복 1번 했을 때 x축 표준편차, y축 승률 그래프(예시를 보여주기 위함)
% - 반복 횟수가 1에서 5로 변할 때 (4, 4, 4, 4, 4)의 승률 그래프

$r=1$일 때를 제외하면 균등 분배 전략의 반복 횟수에 따른 유의미한 승률 차이는 나타나지 않았다. $2 \leq r \leq 5$에서의 균등 분배 전략의 승률 차이는 프로그램의 무작위성에 의한 오차로 판단할 수 있다.

\subsubsection{QRE를 이용한 최적 전략 분석}

III.2.2에서는 상대의 선택이 승률에 선형적으로 비례하는 경우를 가정하였지만, 실제로는 승률에 따른 선택이 완전히 선형적으로 비례한다고 보기 어렵다. 예를 들어 전략 A보다 전략 B를 선택하였을 때 승리할 가능성이 3배 높다면, 대부분의 참가자는 전략 B를 선택할 것이다. 실제 환경에서의 전략 선택은 단순히 승률 값의 비율만을 따르지 않는다. 따라서 다양한 상황에서 참가자가 승률을 자신의 선택에 반영하는 정도가 다를 수 있다는 점을 고려하여야 한다. 이를 보완하기 위해 블로토 대령 게임에 QRE를 적용하여 최적 전략을 분석하였다. I.2.2에서의 QRE 계산 과정을 응용하여 기본적인 형태의 블로토 대령 게임의 QRE를 구하였으며, 그 과정은 아래와 같다.

1. 참가자에게 주어지는 병력의 수와 전장의 수가 주어졌을 때 가능한 모든 병력 배치의 기대 승률을 구한다. 이 값은 곧 기대 보상으로 생각할 수 있다.

2. Logit 함수를 Quantal Response 함수로 사용하여 $i$번째 모델 $M_i$가 병력 배치 $s_j$를 선택할 확률을 결정한다.

\begin{align}
    P_i(s_j)= \frac{e^{\lambda U_{i-1}(s_j)}}{\sum_{}^{} e^{\lambda U_{i-1}(s)}}
\end{align}

위의 식에서 $P_i(s_j)$는 $i$번째 모델이 $s_j$를 선택할 확률, $U_i(s_j)$는 $i$번째 모델에서의 $s_j$의 기대 승률을 의미한다. QRE의 일반적인 계산 과정에서와 마찬가지로 $\lambda$는 음이 아닌 실수 값을 갖는 파라미터이며 0일 경우 완전 무작위, 값이 커질수록 각 병력 배치의 이전 모델에서의 승률을 잘 반영한다.

본 연구에서는 전장의 수 $n=5$, 각 참가자에게 주어지는 병력의 수 $k=20$, 게임 수 $g=25000$로 설정하여 $\lambda$ 값이 0.1, 1, 10, 50으로 증가함에 따라 변화하는 병력 배치 별 승률 경향성을 분석하였다. 이때, 각 $\lambda$ 값마다 Logit 함수를 1$\sim$10번 반복 적용하였다. 아래 표는 $\lambda$ 값과 Logit 함수 반복 횟수 $r$에 따라 승률이 높은 순서대로 병력 배치를 정렬하였을 때, 가장 먼저 등장하는 상위 5개의 배치를 나타낸 것이다. 이때, (4, 4, 4, 3, 5)와 (4, 4, 4, 5, 3)과 같이 순서만 다른 병력 배치의 경우 1개로 세었다.

% //표 삽입(3)
% - 람다 값이 0.1, 1, 10, 50일 때의 Logit 함수 반복 횟수에 따른 상위 5개 조합과 그 승률 변화 표(총 표 개수 4개)

$\lambda=0.1$인 경우, 거의 모든 $r$값에서 균등 분배 전략이 가장 우수한 것으로 나타났다. $r=4$, $r=8$에서 (4, 4, 5, 3, 4) 배치의 승률이 가장 높게 나타났으나 이는 프로그램의 랜덤성에 의한 오차로 생각된다. 이전 모델에서의 승률을 바탕으로 다음 모델에서 어떤 병력 배치를 선택할 확률을 결정하므로, 귀납적으로 생각하였을 때 10보다 큰 $r$에 대해서도 같은 결과가 나타날 것임을 추측할 수 있다.

\vspace{1em}
\begin{table}[htb]
    \centering
    \caption{$\lambda = 0.1$일 때 $r$값에 따른 상위 5개 병력 배치와 승률}
    {\scriptsize
    \begin{tabular}{l|ll|ll|ll|ll|ll}
     & 1위       & 승률      & 2위       & 승률      & 3위       & 승률      & 4위       & 승률      & 5위       & 승률      \\ \hline
    1               & (4,4,4,4,4) & 0.64696 & (4,3,5,4,4) & 0.6412  & (5,5,3,4,3) & 0.63176 & (3,3,4,4,6) & 0.62188 & (4,5,4,2,5) & 0.621   \\
    2               & (4,4,4,4,4) & 0.67044 & (4,4,5,4,3) & 0.6678  & (5,5,3,4,3) & 0.65872 & (3,4,6,3,4) & 0.65096 & (4,4,5,5,2) & 0.65036 \\
    3               & (4,4,4,4,4) & 0.67648 & (4,4,4,5,3) & 0.66964 & (3,5,5,3,4) & 0.65912 & (4,4,6,3,3) & 0.65232 & (4,5,4,5,2) & 0.64992 \\
    4               & (4,4,5,3,4) & 0.6722  & (4,4,4,4,4) & 0.67016 & (3,3,4,5,5) & 0.66232 & (4,6,3,4,3) & 0.6516  & (2,5,4,5,4) & 0.64852 \\
    5               & (4,4,4,4,4) & 0.67552 & (5,4,4,3,4) & 0.67008 & (5,3,5,3,4) & 0.662   & (3,4,3,6,4) & 0.65296 & (4,5,4,5,2) & 0.64812 \\
    6               & (4,4,4,4,4) & 0.68036 & (4,3,4,5,4) & 0.66956 & (4,3,3,5,5) & 0.66164 & (4,3,6,3,4) & 0.65216 & (2,5,4,5,4) & 0.65    \\
    7               & (4,4,4,4,4) & 0.67312 & (4,4,3,5,4) & 0.66888 & (3,3,4,5,5) & 0.66032 & (3,6,3,4,4) & 0.65416 & (4,2,5,4,5) & 0.64992 \\
    8               & (4,4,4,3,5) & 0.67076 & (4,4,4,4,4) & 0.66952 & (4,3,5,3,5) & 0.65968 & (4,3,3,6,4) & 0.65184 & (4,2,5,5,4) & 0.64812 \\
    9               & (4,4,4,4,4) & 0.67356 & (4,4,3,4,5) & 0.67008 & (4,3,5,3,5) & 0.65992 & (4,5,2,4,5) & 0.6498  & (3,3,4,6,4) & 0.64944 \\
    10              & (4,4,4,4,4) & 0.67396 & (4,4,3,4,5) & 0.66972 & (3,5,3,4,5) & 0.66044 & (2,5,5,4,4) & 0.65236 & (3,4,6,4,3) & 0.6502 
    \end{tabular}
    }
    \label{tab:my_label}
\end{table}

$\lambda=1$인 경우 역시 $\lambda=0.1$인 경우와 같이 $r$의 값에 상관없이 항상 균등 분배 전략이 가장 우수하였으나 전체적으로 균등 분배 전략의 승률은 $\lambda=0.1$일 때 보다 낮게 나타났다. 마찬가지로 귀납적인 방법을 통해 10보다 큰 $r$에 대해서도 균등 분배 전략의 승률이 가장 높을 것임을 추측할 수 있다.

\begin{table}[htb!]
    \vspace{1em}
    \centering
    \caption{$\lambda = 1$일 때 $r$값에 따른 상위 5개 병력 배치와 승률}
    {\scriptsize
    \begin{tabular}{l|ll|ll|ll|ll|ll}
       & 1위       & 승률      & 2위       & 승률      & 3위       & 승률      & 4위       & 승률      & 5위       & 승률      \\ \hline
    1  & (4,4,4,4,4) & 0.64668 & (4,3,5,4,4) & 0.64336 & (5,3,3,5,4) & 0.6304  & (5,2,5,4,4) & 0.62176 & (4,4,3,6,3) & 0.62012 \\
    2  & (4,4,4,4,4) & 0.64372 & (4,3,4,5,4) & 0.64048 & (5,3,5,4,3) & 0.63032 & (3,6,3,4,4) & 0.62228 & (5,2,4,5,4) & 0.6192  \\
    3  & (4,4,4,4,4) & 0.64244 & (4,4,5,4,3) & 0.63592 & (4,3,5,5,3) & 0.62884 & (4,6,3,3,4) & 0.62204 & (5,2,4,4,5) & 0.62104 \\
    4  & (4,4,4,4,4) & 0.63864 & (4,5,3,4,4) & 0.6386  & (5,3,4,3,5) & 0.62916 & (2,5,4,4,5) & 0.62232 & (6,3,3,4,4) & 0.6218  \\
    5  & (4,4,4,4,4) & 0.64608 & (4,4,4,5,3) & 0.638   & (3,5,5,4,3) & 0.62704 & (5,4,4,5,2) & 0.62148 & (3,4,4,3,6) & 0.62012 \\
    6  & (4,4,4,4,4) & 0.6484  & (4,4,4,5,3) & 0.63896 & (3,5,3,5,4) & 0.63048 & (3,6,4,4,3) & 0.62248 & (4,4,2,5,5) & 0.62024 \\
    7  & (4,4,4,4,4) & 0.64852 & (4,3,5,4,4) & 0.63964 & (3,3,5,5,4) & 0.63056 & (4,5,4,2,5) & 0.6218  & (3,3,4,4,6) & 0.62044 \\
    8  & (4,4,4,4,4) & 0.64372 & (4,4,3,5,4) & 0.64028 & (5,5,3,4,3) & 0.6292  & (4,4,5,5,2) & 0.6218  & (3,3,4,6,4) & 0.62044 \\
    9  & (4,4,4,4,4) & 0.64364 & (4,3,4,5,4) & 0.6394  & (3,5,3,4,5) & 0.6306  & (3,6,4,4,3) & 0.62004 & (4,4,5,2,5) & 0.61904 \\
    10 & (4,4,4,4,4) & 0.64392 & (4,3,5,4,4) & 0.6386  & (4,5,3,3,5) & 0.63128 & (5,5,4,4,2) & 0.6236  & (6,4,3,4,3) & 0.61956
    \end{tabular}
    }
    \label{tab:my_label}
\end{table}

$\lambda=10$인 경우에는 $r$의 값에 따라 승률이 높은 병력 배치가 다르게 나타났다. $r=1$, 즉 초기 상태에서는 균등 분배 전략의 승률이 가장 높았으나 이후 반복 횟수가 증가함에 따라 (5, 5, 0, 5, 5), (6, 0, 7, 0, 7), (8, 3, 3, 3, 3) 등의 다양한 배치가 상위권에 위치하였다. 이를 통해 이전 모델에서 높은 승률을 보이는 병력 배치를 상대로 승리하는 배치가 다음 모델에서 높은 승률을 가짐을 확인할 수 있다. 그러나 $r$값에 따른 최적 전략의 cycle이 유지되지는 않았으며, $\lambda$값이 더 클 때 cycle이 더욱 명확히 나타날 것으로 예상할 수 있다.
\vspace{9em}
\begin{table}[htb!]
    \centering
    \caption{$\lambda = 10$일 때 $r$값에 따른 상위 5개 병력 배치와 승률}
    {\scriptsize
    \begin{tabular}{l|ll|ll|ll|ll|ll}
       & 1위       & 승률      & 2위       & 승률      & 3위       & 승률      & 4위       & 승률      & 5위       & 승률      \\ \hline
    1  & (4,4,4,4,4) & 0.64416 & (3,5,4,4,4) & 0.63772 & (5,5,3,3,4) & 0.62988 & (5,2,5,4,4) & 0.62376 & (6,3,4,4,3) & 0.6204  \\
    2  & (5,5,0,5,5) & 0.51028 & (0,7,7,6,0) & 0.5016  & (6,4,5,5,0) & 0.49932 & (6,0,4,6,4) & 0.48736 & (5,6,6,3,0) & 0.48296 \\
    3  & (6,0,7,0,7) & 0.46208 & (1,0,7,6,6) & 0.45652 & (6,1,6,6,1) & 0.44744 & (7,6,1,1,5) & 0.43904 & (1,7,0,7,5) & 0.43884 \\
    4  & (8,3,3,3,3) & 0.45748 & (3,3,4,3,7) & 0.45596 & (6,4,4,3,3) & 0.45272 & (7,3,2,4,4) & 0.4512  & (5,6,3,3,3) & 0.44976 \\
    5  & (4,4,4,4,4) & 0.4872  & (4,3,5,4,4) & 0.48672 & (5,0,5,5,5) & 0.48396 & (4,0,5,6,5) & 0.48236 & (5,1,5,5,4) & 0.47804 \\
    6  & (0,0,7,7,6) & 0.46248 & (6,1,0,6,7) & 0.45984 & (1,6,1,6,6) & 0.45628 & (6,6,2,0,6) & 0.45472 & (5,5,0,5,5) & 0.45304 \\
    7  & (0,7,6,7,0) & 0.42932 & (6,2,7,2,3) & 0.4308  & (6,6,7,1,0) & 0.42872 & (1,1,6,6,6) & 0.42756 & (6,2,6,0,6) & 0.4262  \\
    8  & (4,3,5,4,4) & 0.45124 & (3,6,4,3,4) & 0.4496  & (4,4,4,4,4) & 0.44928 & (4,5,5,4,2) & 0.44908 & (4,4,6,4,2) & 0.44556 \\
    9  & (5,5,5,5,0) & 0.45988 & (6,0,5,5,4) & 0.4584  & (4,4,6,6,0) & 0.44988 & (6,0,5,6,3) & 0.44928 & (5,1,5,4,5) & 0.44764 \\
    10 & (6,0,6,7,1) & 0.44304 & (6,6,2,6,0) & 0.44256 & (0,7,6,0,7) & 0.43976 & (6,1,1,6,6) & 0.43968 & (6,1,2,6,5) & 0.4378 
    \end{tabular}}
    \label{tab:my_label}
\end{table} 

$\lambda=50$인 경우에도 $\lambda=10$인 경우와 마찬가지로 $r$값에 따라 승률이 높은 병력 배치가 다르게 나타났다. 초기 상태($r=1$)에서는 균등 분배 전략의 승률이 가장 높았고, $r$값이 증가함에 따라 이전 모델에서 높은 승률을 보이는 병력 배치를 상대로 승리하는 배치가 다음 모델에서 높은 승률을 기록하였다. 표에서 확인할 수 있듯 (7, 0, 6, 7, 0), (2, 1, 8, 8, 1), (4, 4, 3, 3, 6) 등의 병력 배치가 차례로 상위권을 차지하였다. 특히 2개의 전장에 집중한 (2, 1, 8, 8, 1)와 1개의 전장에 더 많은 병력을 투자하고 다른 전장에 균등하게 병력을 분배한 (4, 4, 3, 3, 6)은 $\lambda=10$에서는 나타나지 않았던 전략이다. 표를 통해 $\lambda=50$에서 $\lambda=10$일 때보다 최적 전략의 cycle이 더 잘 유지된다는 점도 확인할 수 있다.

\vspace{1em}
\begin{table}[htb!]
    \centering
    \caption{$\lambda = 50$일 때 $r$값에 따른 상위 5개 병력 배치와 승률}
    {\scriptsize
    \begin{tabular}{l|ll|ll|ll|ll|ll}
       & 1위       & 승률      & 2위       & 승률      & 3위       & 승률      & 4위       & 승률      & 5위         & 승률      \\ \hline
    1  & (4,4,4,4,4) & 0.6436  & (4,5,3,4,4) & 0.64016 & (3,3,5,4,5) & 0.63208 & (4,6,3,4,3) & 0.6212  & (4,5,2,4,5)   & 0.6202  \\
    2  & (7,0,6,7,0) & 0.8434  & (0,8,6,6,0) & 0.77524 & (1,7,6,6,0) & 0.76228 & (5,5,5,5,0) & 0.76048 & (0,2,6,6,6)   & 0.7212  \\
    3  & (2,1,8,8,1) & 0.90636 & (1,1,8,9,1) & 0.90444 & (1,2,7,8,2) & 0.7414  & (1,1,7,9,2) & 0.74132 & (1,1,7,8,3)   & 0.73652 \\
    4  & (4,4,3,3,6) & 1       & (3,5,4,4,4) & 1       & (4,4,4,4,4) & 1       & (6,3,5,3,3) & 1       & (3,3,4,3,7)   & 0.99996 \\
    5  & (5,0,5,5,5) & 0.91008 & (0,5,4,5,6) & 0.83044 & (7,7,0,0,6) & 0.7902  & (6,8,0,0,6) & 0.74576 & (6,6,4,4,0)   & 0.7286  \\
    6  & (7,7,6,0,0) & 0.98052 & (8,0,6,6,0) & 0.9692  & (1,7,6,6,0) & 0.96848 & (0,2,6,6,6) & 0.95172 & (6,1,6,6,1)   & 0.95124 \\
    7  & (2,1,8,8,1) & 0.78352 & (1,8,9,1,1) & 0.7614  & (2,1,8,7,2) & 0.73208 & (1,1,9,7,2) & 0.71124 & (3,1,8,7,1)   & 0.6998  \\
    8  & (4,4,4,4,4) & 0.99988 & (4,4,3,4,5) & 0.99864 & (4,3,3,4,6) & 0.99524 & (3,3,8,3,3) & 0.99492 & (3,3,3,7,4)   & 0.99276 \\
    9  & (0,5,5,5,5) & 0.92984 & (0,5,4,5,6) & 0.85388 & (7,0,0,6,7) & 0.80768 & (6,0,6,0,8) & 0.7658  & (1,5,4,5,5)   & 0.76088 \\
    10 & (0,7,6,7,0) & 0.98508 & (0,8,6,6,0) & 0.97924 & (0,7,6,6,1) & 0.97884 & (0,6,6,6,2) & 0.9682  & (1,6,1,6,6,1) & 0.96044
    \end{tabular}
    }
    \label{tab:my_label}
\end{table}

% //람다가 50일때의 표 여기에 삽입