\section{선행 연구 분석}
블로토 대령 게임은 1921년에 에밀 보렐에 의해 처음 제시되었으며, \cite{borel1953theory} 사회과학 분야에서의 승자 독식 경쟁 상황을 효율적으로 모델링할 수 있다는 장점이 있어 많은 연구가 진행되고 있다. 본 연구에서는 블로토 대령 게임의 변형과 게임 이론의 전략 분석 방법에 관한 다양한 선행 연구를 조사하였다.

\subsection{블로토 대령 게임 관련 선행 연구}
블로토 대령 게임의 기본적인 형태는 2명의 참가자가 있는 제로섬 게임이다. 같은 수의 유한한 병력을 유한한 전장에 배치하여 각 전장에 배치된 병력의 크기 비교로 각 전장의 승패를 도출하고, 이를 통해 게임의 승패를 도출하는 것이다. 따라서 선행 연구에서는 참가자 수 증가와 병력의 비대칭성 부여 등의 과정을 통해 변형 블로토 대령 게임을 새로 정의하고 분석하였다.

블로토 대령 게임의 참가자 수를 증가시킨 연구에서는 각 전장에서 가장 많은 병력을 배치한 한 명의 참가자만이 그 전장에서 승리하는 것으로 정의하였다. 3명의 참가자가 겨루는 대칭적 블로토 대령 게임의 내시 균형을 통해 평형 전략을 계산하는 알고리즘을 구하였다. \cite{boix2020multiplayer}

두 참가자가 게임 시작 시 가지는 병력 수에 비대칭성을 부여한 연구에서는 두 참가자가 갖는 병력 수의 차가 일정 임계값을 기준으로 달라질 때 전략의 새로운 균형이 발생하고 사라짐을 보였다. \cite{roberson2012non}

그러나 조사한 선행 연구 중 각 전장에 가중치를 부여하고, 어떤 전장에 참가자가 배치하는 병력의 효과를 다르게 한 형태의 블로토 대령 게임에 관한 연구는 없는 것으로 파악되었다. 따라서 본 연구는 전장에 가중치를 부여하고, 참가자가 배치하는 병력의 효과를 다르게 한 변형 블로토 대령 게임의 최적 전략을 분석하는 것을 최종적인 목표로 하였다.

\subsection{양자적 반응 균형 관련 선행 연구}

양자적 반응 균형은 기존의 내시 균형에 비해 현실에서 게임 참가자의 선택에 관여하는 다양한 변수를 더욱 잘 반영할 수 있어 게임이론에서 주로 다뤄지는 평형 개념의 하나이다. 본 연구에서는 기존 및 변형 블로토 대령 게임의 최적 확률적 전략을 분석하기 위해 이 개념을 도입하였으며, 그 과정에서 양자적 반응 균형의 적용에 관한 선행 연구를 조사하였다.

죄수의 딜레마와 같이 각 참가자의 선택에 결과가 영향을 받는 게임에서의 QRE를 분석한 연구에서는 게임의 payoff를 행렬표로 나타내었다. 게임의 결과에 따라 각 참가자가 획득하는 보상을 변수로 설정하여 그 값에 따라 평형이 변함을 그래프를 통해 보였다. \cite{mckelvey1995quantal}

본 연구에서는 위와 같은 방식으로 전장의 수와 병력 수가 주어졌을 때 각 병력 배치에 따른 payoff를 계산하고 이를 로짓함수에 대입하여 QRE를 분석하였다. 최적의 확률적 전략을 도출하여 이를 그래프와 표로 시각화함으로써 본 연구의 결과를 더욱 명확하게 표현하였다.