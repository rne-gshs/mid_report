\section{결론 및 향후 연구}

본 연구에서는 블로토 대령 게임의 규칙을 수학적으로 정의하고 상대의 병력 배치 선택 방법에 따라 경우를 나누어 최적 전략을 분석하였다. 상대의 선택이 무작위적으로 이뤄지는 경우 각 전장에 균등하게 병력을 배치하는 균등 분배 전략이 가장 높은 승률을 보였으며, 참가자에게 주어진 병력의 수가 증가할수록 균등 분배 전략의 승률은 증가하였다. 이로부터 병력 배치 방법의 경우의 수, 즉 무질서도가 증가할수록 균등 분배 전략은 더욱 안정적임을 알 수 있다. 상대의 선택이 각 전략의 승률에 선형적으로 비례하는 경우 역시 균등 분배 전략이 가장 우수하였으며, 승률도 안정적으로 유지되었다.

위의 두 경우에서는 다양한 상황에 따라 참가자가 자신의 선택에 승률을 반영하는 정도가 다를 수 있다는 점을 고려하지 않았다. 이를 보완하고자 양자적 반응 균형(QRE)를 적용하였으며, 전략의 승률을 반영하는 정도를 나타내는 파라미터인 $\lambda$의 값을 다르게 하여 각 경우에 대해 분석하였다. 그 결과 $\lambda$가 0.1, 1과 같이 작을 때는 항상 균등 분배 전략의 승률이 가장 높았지만 10, 50으로 그 값이 커짐에 따라 여러 병력 배치가 상성 cycle을 형성하였다. 즉, 게임의 참가자가 전략 별 승률에 민감하게 반응할수록 상성을 잘 파악하고 대응한다는 결론을 내릴 수 있다.

본 연구는 기존에 고려되지 않았던 다양한 전략 결정 방식에 대해 기본 블로토 대령 게임의 확률적 전략을 분석하여 참가자의 선택을 이해하기 위한 수학적$\cdot$게임이론적 기반을 마련하였다는 점에서 의의를 갖는다. 본 연구를 바탕으로 추후 전장마다 서로 다른 가중치를 부여하고, 전장 별 각 참가자의 병력 효율을 고려한 확장된 블로토 대령 게임의 최적 전략을 연구하고자 한다. 이후 연구 결과를 바탕으로 승자 독식 선거와 같은 사회 현상에 반영하여 블로토 대령 게임의 실제 적용 가능성을 확인할 것이다.