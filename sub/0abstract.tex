\begin{abstract}
1921년 에밀 보렐이 처음 제시한 블로토 대령 게임(Colonel Blotto Game)은 게임 이론에서의 주요 문제 중 하나로, 기본적으로 두 참가자가 제한된 병력을 여러 전장에 분배하여 승패를 결정짓는 게임이다. 이는 사회과학 분야에서 예산이 제한된 승자 독식 경쟁을 분석하는데 널리 사용되었으며, 그 예시로는 국가 간 전쟁과 선거, 광고, 경매 등이 있다. 본 연구에서는 블로토 대령 게임의 규칙과 진행 과정을 수학적으로 정의하였다. 이를 바탕으로 블로토 대령 게임을 C++ 프로그래밍을 통해 구현하였고, 내시 균형의 확장인 양자적 반응 균형(Quantal Response Equilibrium, 이하 QRE)을 적용하여 기본 블로토 대령 게임에서 최적의 확률적 전략을 분석하였다. 추후 서로 다른 전장의 중요도에 따른 가중치를 부여하거나 각 참가자의 전장 별 병력 효율을 다르게 하는 등의 확장된 블로토 대령 게임에 관한 연구를 이어갈 것이다.
\end{abstract}