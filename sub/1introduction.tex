\section{서론}
\subsection{연구 목적 및 필요성}
게임이론은 상호 의존적이고 이성적인 의사 결정에 대한 수학적 이론이다. 개인이나 기업이 어떤 행위를 했을 때, 그 결과가 자신 뿐만 아니라 다른 참가자의 행동에 의해서도 결정되는 상황에서, 자신의 최대 이익에 부합하는 행동을 추구한다는 수학적 이론을 연구한다.
블로토 대령 게임(Colonel Blotto Game)은 게임이론에서 연구되는 주요 문제 중 하나로, 두 참가자가 각각 제한된 병력을 여러 전장에 분배하여 각 전장에서 더 많은 병력을 배치한 참가자가 해당 전장에서 승리하며, 최종적으로 더 많은 전장에서 승리한 참가자가 승리하는 게임이다. 이러한 블로토 대령 게임은 국가간의 전쟁, 선거, 광고 투자, 경매와 같은 현실의 다양한 사회 현상에 적용할 수 있다.
블로토 대령 게임에 대한 대부분의 선행연구에서 서로 다른 전장의 중요도는 고려하지 않았다. 그러나 선거에서 여러 지역의 선거구의 중요도가 모두 같을 수는 없는 것처럼, 현실의 사회 현상을 분석할 때 각 전장의 중요도를 고려할 필요가 있다.
따라서 본 연구에서는 우선적으로 기존의 블로토 대령 게임을 분석하고, 이후 전장의 중요도에 따른 가중치를 부여한 확장된 블로토 대령 게임의 최적화된 전략을 도출할 것이다.
\subsection{이론적 배경}
\subsubsection{블로토 대령 게임(Colonel Blotto Game)}
먼저, 블로토 대령 게임의 규칙과 진행 방법을 수학적으로 정의하였다. 기본적인 형태의 블로토 대령 게임은 두 참가자가 각각 $k$명의 병력을 가진 


$a + b = c^2$

$_{n}\mathrm{C}_{k}$